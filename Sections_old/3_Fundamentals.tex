

\section{Preliminaries}\label{sec3}

%http://www.math.unipd.it/~frossi/chapter4.pdf  Francesca Rossi, Peter van Beek, Toby Walsh


\textit{Constraint programming} is a paradigm for solving combinatorial problems where relationships between variables are expressed in the form of constraints \citep{rbw06}.
A constraint problem is solved when all the variables have been instantiated and the constraints are satisfied. In general, variables range over a finite set of values, corresponding with \textit{Constraint Programming over Finite Domains} \cite{schulte2006finite}.


\begin{definition}
  A \textit{Constraint Satisfaction Problem}
  \citep{guns2017miningzinc} (CSP) is defined as a triple
  $( X, D, C)$, where $X=(x_{1},\dots ,x_{n})$ is a tuple of $n$
  variables.  $D=D_{1}\times D_{2}\cdots \times D_{n}$, such that
  $D_i$ is the domain of values for each variable $x_i$, and
  $C=(c_{1},\dots ,c_{m})$ are the restrictions that delimit the
  possible values that the variables can take.  Each constraint
  $c_{i}$ is a subset of $D$ that indicates the values that satisfy
  the restriction.
\end{definition}

%An \emph{assignment} of the variables is a function from a subset of variables to a particular set of values in the corresponding subset of domains.
%An assignment  satisfies a constraint $(R,X')$ if the values assigned to the variables belonging to $X'$ satisfy the relation $R$.
%An assignment is a \emph{solution} of a CSP if it does not violate any of the constraints and includes all variables.
%If a CSP has at least one solution, then the CSP is \emph{satisfiable}; otherwise, the CSP is \emph{inconsistent}.


Specific constraint solvers are used to find the solution of CSP problems, that is, a \textit{constraint solver} is a decision procedure that checks whether a set of constraints can be satisfied.
Constraint solvers perform a systematic exploration of the search space until either a solution is found or the whole tree is explored without success.
%In order to reduce the search space,  {\em constraint  propagation} techniques are applied to reduce the domains of variables~\citep{chls06}.
%The application of the solver implies that either the first solution is found or the unsatisfiability of the problem is proven before.
%
Solvers uses classical propagation techniques \citep{van89} combined with {\em backtracking},
a recursive algorithmic that tries to build a solution incrementally, that is, if the partial assignment of the variables is not suitable, then backtrack and try other assignments.


MiniZinc~\cite{DBLP:journals/constraints/MarriottNRSBW08} is a modeling
language that helps to formulate a CSP. The user can specify a CSP by
using typed variables (Boolean, Integer, Float, Set, Array) and
expressions that can use mathematical expressions (sums, products,
comparison, logical operators), looping over sets (forall, exists), etc.
A MiniZinc model can be seen as a program that solve a constraint
problem. These programs (or models) the objects we are going to
consider for testing: Implementation Under Testing (IUT).

Scheduling problems are modeled with certain types of constraints.
Specifically, we deal with the following constraints in this work:
\textit{precedence constraints} where it is specified that a constraint
must be executed before another; and the \textit{resource
  limitations constraints} where certain tasks need to use a series of
resources which are limited and shared. In order to formalize a
scheduling problem we need to consider a finite set of \emph{tasks} that we will
denote $T$, we will denote the elements of $T$ as $t_{1}$ , $t_{2}$,
etc.
We can have one or more processors where the tasks are executed, so we
need a finite set of processors $P$. Some problems do not consider
processors, in these cases we consider that there is only one processor.
Each individual task will have a \emph{duration} that depends on the
processor it is executed, we will
consider a function $D$ that will indicate the duration of any task, a
positive real number.
We will consider an order among the tasks $\prec$ that
will indicate the precedence between tasks: $t_{i}\prec t_{j}$.
Each task need some resources, for instance in the \emph{Moving
  Furniture Problem}, moving a piano needs
3 people and 2 trolleys. Therefore, we need to consider a multiset of
Resources that we will denote $R$,
the multiplicity of any element denotes the number of
instances available for each resource. Then, we need to consider the
resources needed for completing a task, it will be represented by a
function $L$: $L(t)$ is the multiset of resources needed for the task
$t$. Finally, the whole scheduling problem must be completed before a
fixed deadline that we denote $\dln$.

\begin{definition}
  A \emph{scheduling problem instance} is a tuple $i=(T, P, D, \prec, R, L, \dln)$, where
  \begin{itemize}
  \item $T$ is a finite set of tasks. We can assume a numeration of
    the tasks $T=\{t_{1},\ldots t_{n}\}$. For the shake of simplicity
    we will call $T(k)=t_{k}$, for $1\leq k\leq n$.
  \item $P$ is a finite set of processors.
  \item $D$ is a mapping $D:T\times P\mapsto \real^{+}$
    \footnote{$\real^{+}=\{x\ |\ x\in\real,\ x>0\}$}
  \item $\prec\subseteq T\times T$ is a non-reflexive and transitive relation that
    indicates the precedence among tasks.
  \item $R$ is a multiset of resources.
  \item $L: T\mapsto\calP(R)$ is the mapping of associates the
    resources needed for any task.
  \item $\dln\in\real^{+}$ is the deadline of the scheduling problem.
  \end{itemize}
  We denote the set of scheduling problem instances as \SPI.
\end{definition}


Scheduling problem instances are the inputs of the model implemented
in MiniZinc. When executing the model with a particular instance, the
output of MiniZinc gives the minimum time require to complete all the
tasks and schedule.
In this paper, we are going to
consider only the minimum time in which it is possible to
schedule the problem, that is the \emph{makespan}. Sometimes the
scheduling problem has not a solution. In this case, we consider that
the makespan is infinite. To formalize this, we use
the symbol $\infty$. We will denote
$\sol(i)\in\real^{+}\cup\{\infty\}$\footnotemark the solution of the scheduling
problem $S$.

\footnotetext{We assume $x<\infty$ for all $x\in\real$ and $\infty\not<\infty$}

\begin{definition}
Let us consider a scheduling problem.
A \emph{metamorphic relation}  $MR\subseteq\SPI\times\SPI\times(\real^{+}\cup\{\infty\})\times(\real^{+}\cup\{\infty\})$
is a relation over two
scheduling problem instances $i_1$ and $i_2$, and the outputs
corresponding
to the execution of $o_{1}=\sol(i_1)$ and $o_{2}=sol(i_2)$.
$i_1$ is a \emph{source input} and that $i_2$ is a \emph{follow-up input}.

% \begin{framed}
% \begin{center}

% $MR(i_1,i_2,o_{1},o_{2})$

% $ \Updownarrow_{def}$

% $\forall i_{1}, i_{2}, o_{1}, o_{2}: C(i_1, i_2, o_{1}, o_{2})$
% \end{center}
% \end{framed}
% Where $C$ is a condition whose free variables belong to the set $\{i_{1}, i_{2}, o_{1}, o_{2}\}$,
\end{definition}



%%% Local Variables:
%%% mode: latex
%%% TeX-master: "../MT_scheduling"
%%% End:
