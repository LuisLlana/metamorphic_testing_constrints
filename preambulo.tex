%%
%% Copyright 2007-2020 Elsevier Ltd
%%
%% This file is part of the 'Elsarticle Bundle'.
%% ---------------------------------------------
%%
%% It may be distributed under the conditions of the LaTeX Project Public
%% License, either version 1.2 of this license or (at your option) any
%% later version.  The latest version of this license is in
%%    http://www.latex-project.org/lppl.txt
%% and version 1.2 or later is part of all distributions of LaTeX
%% version 1999/12/01 or later.
%%
%% The list of all files belonging to the 'Elsarticle Bundle' is
%% given in the file `manifest.txt'.
%%

%% Template article for Elsevier's document class `elsarticle'
%% with numbered style bibliographic references
%% SP 2008/03/01
%%
%%
%%
%% $Id: elsarticle-template-num.tex 190 2020-11-23 11:12:32Z rishi $
%%
%%
\documentclass[final,twocolumn]{elsarticle}
%\documentclass[review,twocolumn, 3p]{elsarticle}  %Sonia ¿Qué es 3p?
\usepackage[utf8]{inputenc}
%\documentclass{cas-dc}

%% Use the option review to obtain double line spacing
%% \documentclass[authoryear,preprint,review,12pt]{elsarticle}

%% Use the options 1p,twocolumn; 3p; 3p,twocolumn; 5p; or 5p,twocolumn
%% for a journal layout:
%% \documentclass[final,1p,times]{elsarticle}
%% \documentclass[final,1p,times,twocolumn]{elsarticle}
%% \documentclass[final,3p,times]{elsarticle}
%% \documentclass[final,3p,times,twocolumn]{elsarticle}
%% \documentclass[final,5p,times]{elsarticle}
%% \documentclass[final,5p,times,twocolumn]{elsarticle}

%% For including figures, graphicx.sty has been loaded in
%% elsarticle.cls. If you prefer to use the old commands
%% please give \usepackage{epsfig}
%% The amssymb package provides various useful mathematical symbols
\usepackage{amssymb}
%% The amsthm package provides extended theorem environments
\usepackage{amsthm}

%% The lineno packages adds line numbers. Start line numbering with
%% \begin{linenumbers}, end it with \end{linenumbers}. Or switch it on
%% for the whole article with \linenumbers.
%% \usepackage{lineno}
\usepackage{verbatim}
\usepackage{framed}

\usepackage{caption}
\usepackage{subcaption}
\usepackage{hyperref}

\usepackage{listings}  % code
\usepackage{tikz}
\usetikzlibrary{arrows,backgrounds,calc,positioning}



%\newcolumntype{d}[1]{D{.}{.}{#1}}
\tikzstyle{obj} = [
%draw=black,
%      line width=0.5pt,
      anchor=center,
      text centered,
      rounded corners,
%         fill=blue!20,
%         minimum width=0.5cm,
          minimum height=3mm,
      text badly centered
      ]
\tikzstyle{ar} = [->, bend left]


\tikzset{
    mynode/.style={rectangle,rounded corners,draw=black, top color=white, bottom color=yellow!50,very thick, inner sep=1em,
    inner ysep=8pt, % El ancho de las cajitas
    text centered, scale=0.8},
    mynodeRed/.style={rectangle,rounded corners,draw=black, top color=red!50, bottom color=red!20,very thick, inner sep=1em,
    inner ysep=8pt, % El ancho de las cajitas
    %minimum size=3.5em,
    text centered, scale=0.8},
    mynodeGreen/.style={rectangle,rounded corners,draw=black, top color=white, bottom color=yellow!50,very thick, inner sep=1em,
    inner ysep=8pt, % El ancho de las cajitas
    text centered, scale=0.8},
    myarrow/.style={->, >=latex', shorten >=2pt},
    mylabel/.style={text centered, fill=blue!20,scale=0.8}
}

%\lstset{
%    breaklines=true,
%    basicstyle=\footnotesize
%}





\usepackage{tikz}
\usetikzlibrary{shapes.multipart,chains,positioning,arrows,calc,arrows.meta}



\newtheorem{theorem}{Theorem}
\newtheorem{proposition}[theorem]{Proposition} %% remove [theorem] for Unique numbering
%\theoremstyle{plain}
\newtheorem{example}[theorem]{Example} %% remove [theorem] for Unique numbering
\newtheorem{definition}[theorem]{Definition} %% remove [theorem] for Unique numbering






\newcommand{\unsat}[1] {\texttt{unSat(#1)}}
\newcommand{\sat}[1] {\texttt{sat(#1)}}



%Numbered environment defined with Newtheorem
\usepackage{amsmath}

\newtheorem{ResearchQuestion}{Research Question}[]




%----------------------------------------------%
% Syntax highlighting for MiniZinc in listings %
%----------------------------------------------%

\definecolor{lightgray}{rgb}{0.97, 0.97, 0.97}
\definecolor{ForestGreen}{RGB}{34,106,46}

\lstdefinelanguage{minizinc}{
    morekeywords={
        %% MiniZinc keywords
        %%
        ann, annotation, any, array, assert,
        bool,
        constraint,
        else, elseif, endif, enum, exists,
        float, forall, function,
        if, in, include, int,
        list,
        minimize, maximize,
        of, op, output,
        par, predicate,
        record,
        set, solve, string,
        test, then, tuple, type,
        var,
        where,
        <, <=, >, >=,
        +,-,
        %% MiniZinc functions
        %%
        abort, abs, acosh, array_intersect, array_union,
        array1d, array2d, array3d, array4d, array5d, array6d, asin, assert, atan,
        bool2int,
        card, ceil, combinator, concat, cos, cosh,
        dom, dom_array, dom_size, dominance,
        exp,
        fix, floor,
        index_set, index_set_1of2, index_set_2of2, index_set_1of3, index_set_2of3, index_set_3of3,
        int2float, is_fixed,
        join,
        lb, lb_array, length, let, ln, log, log2, log10,
        min, max,
        pow, product,
        round,
        set2array, show, show_int, show_float, sin, sinh, sqrt, sum,
        tan, tanh, trace,
        ub, and ub_array,
        %% Search keywords
        %%
        bool_search, int_search, seq_search, priority_search,
        %% MiniSearch keywords
        %%
        minisearch, search, while, repeat, next, commit, print, post, sol, scope, time_limit, break, fail
    },
    sensitive=true, % are the keywords case sensitive
    morecomment=[l][\em\color{ForestGreen}]{\%},
    %morecomment=[s]{/*}{*/},
    morestring=[b]",
}

%% Settings for listings
%%
\lstset{ %
%    backgroundcolor=\color{lightgray},  % choose the background color; you must add
                                        % \usepackage{color} or \usepackage{xcolor}
    %basicstyle=\ttfamily\small,    % the size of the fonts that are used for the code
    basicstyle=\small,    % the size of the fonts that are
    aboveskip=3mm,
    belowskip=1mm,
    breakatwhitespace=false,            % sets if automatic breaks should only happen at whitespace
    breaklines=true,                    % sets automatic line breaking
    captionpos=b,                       % sets the caption-position to bottom
    commentstyle=\color{ForestGreen},   % comment style
    %deletekeywords={...},              % if you want to delete keywords from the given language
    escapeinside={\%*}{*)},             % if you want to add LaTeX within your code
    extendedchars=true,                 % lets you use non-ASCII characters; for 8-bits
                                        % encodings only, does not work with UTF-8
%    frame=single,                       % adds a frame around the code
    keepspaces=true,                    % keeps spaces in text, useful for keeping indentation
                                        % of code (possibly needs columns=flexible)
    keywordstyle=\rmfamily\color{blue}, % keyword style
%   keywordstyle=\bfseries\color{blue}, % keyword style
    language=minizinc,                  % the language of the code
    %morekeywords={*,...},              % if you want to add more keywords to the set
    numbers=none,                       % where to put the line-numbers; possible values are (none, left, right)
    %numbersep=5pt,                     % how far the line-numbers are from the code
    %numberstyle=\tiny\color{Gray},     % the style that is used for the line-numbers
    rulecolor=\color{black},            % if not set, the frame-color may be changed
                                        % on line-breaks within not-black text (e.g. comments (green here))
    showspaces=false,                   % show spaces everywhere adding particular
                                        % underscores; it overrides 'showstringspaces'
    showstringspaces=false,             % underline spaces within strings only
    showtabs=false,                     % show tabs within strings adding particular underscores
    %stepnumber=1,                      % the step between two line-numbers. If it's 1, each line will be numbered
    stringstyle=\color{Red},            % string literal style
    tabsize=2,                          % sets default tabsize to 2 spaces
    title=\lstname                      % show the filename of files included with \lstinputlisting;
                                        % also try caption instead of title
}



\newcommand{\dln}{\ensuremath{\mathit{dln}}}


\newcommand{\calP}{{\cal P}}
\newcommand{\real}{\mathbb{R}}
\newcommand{\sol}{\mathsf{sol}}

\def\luisSays#1{{\color{red} Comentario Luis: #1}}
\newcommand{\SPI}{\ensuremath{\mathit{SPI}}}
%%% Local Variables:
%%% mode: latex
%%% TeX-master: "MT_scheduling"
%%% End:
