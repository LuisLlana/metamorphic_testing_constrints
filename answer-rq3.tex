\subsection{\rqidentifying: features of mutation-effective metamorphic relations}
The definition and explanation of the metamorphic rules are in
Section~\ref{sec:def_MT}. In this section, we examine how each
metamorphic rule affects each part of the \lstinline|rcpsp.mzn| listed
in~\ref{lst:minizinc-rcpsp}. The constraints of the programs are
divided into four groups:

\begin{enumerate}
\item\label{item:precdence} Precedence constraints.
\item\label{item:redundant} Redundant non-overlapping constraints.
\item\label{item:cumulative} Cumulative resource constraints.
\item\label{item:makespan} Makespan constraints.
\end{enumerate}

Metamorphic rule \mr{1} detects defects in all groups except
cumulative resource constraints; \mr{2} detects defects in precedence
constraints; and \mr{3} and \mr{4} detect defects in all groups.

The metamorphic rules detect the 6 mutants that are not identified
using only the mutation operators. The most effective rule is \mr{3},
which detects 63 mutants out of 74 mutants,
followed by \mr{4}, which detects 48
mutants, \mr{1}, which detects 24 mutants, and \mr{2}, which detects
six mutants. All the mutants detected by \mr{2} are also detected by
the other rules. Additionally, there are two mutants detected by both \mr{1}
and \mr{4} that are not detected by \mr{3}.
%%% Local Variables:
%%% mode: LaTeX
%%% TeX-master: "MT_scheduling"
%%% End:
