\newpage
\subsection{Result and Discussion/ Experimental results}

% [!htb]
\begin{table*}[htb]
    \centering
    \begin{tabular}{l|c|c|c}

    j30\_10\_01  &  rcpsp.dzn  & fu-cycle.dzn & fu-all-prec.dzn\\
    makespan  & (original) &  & \\
    \hline
     rcpsp.mzn    & 41 & UNSAT & 157 \\
     (original)   &  &  & \\
     \hline
    
    MUT$\leq$-$<$.mzn & 45 & UNSAT & UNSAT \\
     \hline
    
    MUTforall-exists  & 6 & 0& 6 \\
     \hline
    \end{tabular}
    \caption{rows: codes (mutations); columns: data (follow-ups). UNSAT is the sort-name of UNSATISFIABLE }
    \label{tab:outs}
\end{table*}

En la tabla \ref{tab:outs} hay que comparar:
\begin{itemize}
    \item MR1: Si se ponen todas las precedencias en secuencial (fu-all-prec.dzn) entonces el makespan (157) debe ser finito y mayor o igual al original rcpsp.dzn (41). Esto es lo que ocurre al ejecutar el código original (rcpsp.mzn).

    El mutante MUT$\leq$-$<$.mzn se puede matar porque no es finito.

    El mutante MUTforall-exists no se puede matar.

    \item MR2: Si se crea un ciclo con las precedencias (fu-cycle.dzn), entonces debe ser UNSATISFIABLE. Esto es lo que ocurre al ejecutar el código original (rcpsp.mzn).

    El mutante MUT$\leq$-$<$.mzn no se puede matar.

    El mutante MUTforall-exists se puede matar pues tiene un resultado finito (0).

\end{itemize}
