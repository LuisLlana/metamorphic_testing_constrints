

\onecolumn


\section{Car sequencing problem}

Este problema tiene mucha literatura.
En el ejemplo se cita a \cite{regin1997filtering} y este trabajo cita a
\cite{dincbas1988solving}.

Descripción: \textit{ "A number of cars are to be produced; they are not identical, 
because different options are available as variants on the basic model. 
 The assembly line has different stations which install the various options (air-conditioning, sun-roof, etc.). 
 These stations have been designed to handle at most a certain percentage of 
 the cars passing along the assembly line. 
 Furthermore, the cars requiring a certain option must not be bunched together, 
 otherwise the station will not be able to cope. Consequently, the cars must be 
 arranged in a sequence so that the capacity of each station is never exceeded. 
 For instance, if a particular station can only cope with at most half of the 
 cars passing along the line, the sequence must be built so that at most 1 car in any 2 requires that option."}

\begin{figure}[!htb]
\begin{center}
\begin{tikzpicture}

 \node [mynode](I) at (-3.8,1) {$i_{1}$ (\lstinline|car_test_0.dzn|) };
 \node [mynode](O) at (-3.8,-1.1) {$o_{1}<\infty$};

 \node [mylabel](code) at (0,-0.3) {\lstinline|carseq.mzn|};
 \node [mynode](I') at (3.8,1) {$i_{2}$ (\lstinline|car_test_0-fu-k.dzn|)};
 \node [mynode](O') at (3.8,-1.1) {$o_{2} =\infty$};

 \draw[myarrow](I) -- (O);
 \draw[myarrow](I') -- (O');

 \node [mynode,below= -0.2cm of I] (Icaption) {
\begin{tabular}{c}
%    n\_cars =  12\\
    k = [2, 3, 3, 5, 5]
\end{tabular}
  };
  
% \node [mynode,below= -0.2cm of I] (Icaption) {n\_cars =  12};
% \node [mynode,below= -0.5cm of I] (Icaption2) {n\_cars\_by\_confs = [4, 1, 1, 2, 1, 1, 2]};
 \node [mynode,below= -0.2cm of I'] (I'caption) {k = [5, 5, 5, 5, 5]};

\end{tikzpicture}
\subcaption{Original program execution}
\label{Fig:MR4 carseq-con-mas-handlers-a}
\end{center}

\begin{center}
\begin{tikzpicture}
 \node [mynode](I) at (-4,1) {$i_{1}$ (\lstinline|car_test_0.dzn|)};
 \node [mylabel](code) at (0,-0.3) {\lstinline|carseqMUT-forall-exists.mzn|};
 \node [mynode](I') at (4,1) {$i_{2}$ (\lstinline|car_test_0-fu-k.dzn|)};

 \node [mynode](O) at (-4,-1.2) {$o_{1} <\infty$};
 \node [mynodeRed](O') at (4,-1.2) {$o_{2}<\infty$};

 \draw[myarrow](I) -- (O);
 \draw[myarrow](I') -- (O');

 \node [mynode,below= -0.2cm of I] (Icaption) {k = [2, 3, 3, 5, 5]};
 \node [mynode,below= -0.2cm of I'] (I'caption) {k = [5, 5, 5, 5, 5]};
\end{tikzpicture}
\subcaption{Mutation execution}
\label{Fig:MR4 carseq-con-mas-handlers-b}
\end{center}
\caption{Metamorphic relation MR4 and mutation execution}
\label{Fig:MR4 carseq-con-mas-handlers}
\end{figure}


¿¿ Qué es  \texttt{k} ??

Veamos un ejemplo con los datos \texttt{car\_test\_0.dzn}:

\begin{verbatim}
Problem description: 
	 number of cars = 10
	 number of options =  5
	 number of configurations =  7
	 k =  [2, 3, 3, 5, 5]
	 min capacity =  [0, 0, 0, 0, 0]
	 max capacity =  [1, 2, 1, 2, 1]
	 number of cars in configurations  =  [2, 1, 1, 2, 1, 1, 2]
	 confs  = [1, 0, 0, 1, 0, 
           0, 0, 0, 1, 0, 
           0, 1, 0, 0, 0, 
           1, 0, 0, 0, 0, 
           0, 0, 0, 0, 1, 
           0, 1, 0, 1, 0, 
           0, 0, 1, 0, 0]
 
Solution: 
	 cars = [1, 6, 4, 7, 5, 2, 7, 4, 3, 1]
	 options by cars = [1, 0, 1, 0, 0, 0, 0, 1, 0, 1, 
                    0, 1, 0, 0, 0, 0, 0, 0, 1, 0, 
                    0, 0, 0, 1, 0, 0, 1, 0, 0, 0, 
                    1, 1, 0, 0, 0, 1, 0, 0, 0, 1, 
                    0, 0, 0, 0, 1, 0, 0, 0, 0, 0]
\end{verbatim}


\texttt{k} es el tamaño de la secuencia de coches que se requiere para no colapsar las estaciones. Es decir,
en la matriz de \texttt{options by cars}, para 
cada subsecuencia de tamaño  k =  [2, 3, 3, 5, 5] la suma de sus números (0/1) debe ser \texttt{>= (min capacity =  [0, 0, 0, 0, 0])} y   \texttt{ <= (max capacity =  [1, 2, 1, 2, 1])}.

Por lo tanto el follow-up hay que construirlo de tal forma que si el tamaño de k o secuencia de coches, es mayor que el número de coches entonces podemos aplicar la MR3.

Desglosado:
\begin{itemize}
    \item En la primera fila de la matriz, la suma de cada subsecuencia de tamaño 2 es $<=$ 1.
    \item En la segunda fila de la matriz, la suma de cada subsecuencia de tamaño 3 es $<=$ 2.
    \item En la tercera fila de la matriz, la suma de cada subsecuencia de tamaño 3 es $<=$ 1.
    \item En la cuarta fila de la matriz, la suma de cada subsecuencia de tamaño 5 es $<=$ 2.
    \item En la quinta fila de la matriz, la suma de cada subsecuencia de tamaño 5 es $<=$ 1.
\end{itemize}


Observación: si ejecuto el mutante con \texttt{k =  [5, 5, 5, 5, 5]} o \texttt{k =  [6, 6, 6, 6, 6]}, 
entonces $i_1$ es UNSATISFIABLE y el foolow-up $i_2$ es una solucióncualquiera, cuando debería ser insatisfactible.

Si aunmento a \texttt{k = [7, 7, 7, 7, 7]}, entonces son UNSATISFIABLE tanto $i_1$ como el foolow-up $i_2$.

\medskip
\medskip

Probar con el resto de test  disponibles en 
\url{https://github.com/MiniZinc/minizinc-benchmarks}


\section{Carpet cutting problem}

Tarda demasiado tiempo,... pienso que hay algo mal, quizás es otro resolutor,... esun ejercicio antiguo,  