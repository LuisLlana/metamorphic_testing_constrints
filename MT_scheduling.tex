%%
%% Copyright 2007-2020 Elsevier Ltd
%%
%% This file is part of the 'Elsarticle Bundle'.
%% ---------------------------------------------
%%
%% It may be distributed under the conditions of the LaTeX Project Public
%% License, either version 1.2 of this license or (at your option) any
%% later version.  The latest version of this license is in
%%    http://www.latex-project.org/lppl.txt
%% and version 1.2 or later is part of all distributions of LaTeX
%% version 1999/12/01 or later.
%%
%% The list of all files belonging to the 'Elsarticle Bundle' is
%% given in the file `manifest.txt'.
%%

%% Template article for Elsevier's document class `elsarticle'
%% with numbered style bibliographic references
%% SP 2008/03/01
%%
%%
%%
%% $Id: elsarticle-template-num.tex 190 2020-11-23 11:12:32Z rishi $
%%
%%
\documentclass[final,twocolumn]{elsarticle}
%\documentclass[review,twocolumn, 3p]{elsarticle}  %Sonia ¿Qué es 3p?
\usepackage[utf8]{inputenc}
%\documentclass{cas-dc}

%% Use the option review to obtain double line spacing
%% \documentclass[authoryear,preprint,review,12pt]{elsarticle}

%% Use the options 1p,twocolumn; 3p; 3p,twocolumn; 5p; or 5p,twocolumn
%% for a journal layout:
%% \documentclass[final,1p,times]{elsarticle}
%% \documentclass[final,1p,times,twocolumn]{elsarticle}
%% \documentclass[final,3p,times]{elsarticle}
%% \documentclass[final,3p,times,twocolumn]{elsarticle}
%% \documentclass[final,5p,times]{elsarticle}
%% \documentclass[final,5p,times,twocolumn]{elsarticle}

%% For including figures, graphicx.sty has been loaded in
%% elsarticle.cls. If you prefer to use the old commands
%% please give \usepackage{epsfig}
%% The amssymb package provides various useful mathematical symbols
\usepackage{amssymb}
%% The amsthm package provides extended theorem environments
\usepackage{amsthm}

%% The lineno packages adds line numbers. Start line numbering with
%% \begin{linenumbers}, end it with \end{linenumbers}. Or switch it on
%% for the whole article with \linenumbers.
%% \usepackage{lineno}
\usepackage{verbatim}
\usepackage{framed}

\usepackage{caption}
\usepackage{subcaption}
\usepackage{hyperref}

\usepackage{listings}  % code
\usepackage{tikz}
\usetikzlibrary{arrows,backgrounds,calc,positioning}



%\newcolumntype{d}[1]{D{.}{.}{#1}}
\tikzstyle{obj} = [
%draw=black,
%      line width=0.5pt,
      anchor=center,
      text centered,
      rounded corners,
%         fill=blue!20,
%         minimum width=0.5cm,
          minimum height=3mm,
      text badly centered
      ]
\tikzstyle{ar} = [->, bend left]


\tikzset{
    mynode/.style={rectangle,rounded corners,draw=black, top color=white, bottom color=yellow!50,very thick, inner sep=1em,
    inner ysep=8pt, % El ancho de las cajitas
    text centered, scale=0.8},
    mynodeRed/.style={rectangle,rounded corners,draw=black, top color=red!50, bottom color=red!20,very thick, inner sep=1em,
    inner ysep=8pt, % El ancho de las cajitas
    %minimum size=3.5em,
    text centered, scale=0.8},
    mynodeGreen/.style={rectangle,rounded corners,draw=black, top color=white, bottom color=yellow!50,very thick, inner sep=1em,
    inner ysep=8pt, % El ancho de las cajitas
    text centered, scale=0.8},
    myarrow/.style={->, >=latex', shorten >=2pt},
    mylabel/.style={text centered, fill=blue!20,scale=0.8}
}

%\lstset{
%    breaklines=true,
%    basicstyle=\footnotesize
%}





\usepackage{tikz}
\usetikzlibrary{shapes.multipart,chains,positioning,arrows,calc,arrows.meta}



\newtheorem{theorem}{Theorem}
\newtheorem{proposition}[theorem]{Proposition} %% remove [theorem] for Unique numbering
%\theoremstyle{plain}
\newtheorem{example}[theorem]{Example} %% remove [theorem] for Unique numbering
\newtheorem{definition}[theorem]{Definition} %% remove [theorem] for Unique numbering






\newcommand{\unsat}[1] {\texttt{unSat(#1)}}
\newcommand{\sat}[1] {\texttt{sat(#1)}}



%Numbered environment defined with Newtheorem
\usepackage{amsmath}

\newtheorem{ResearchQuestion}{Research Question}[]




%----------------------------------------------%
% Syntax highlighting for MiniZinc in listings %
%----------------------------------------------%

\definecolor{lightgray}{rgb}{0.97, 0.97, 0.97}
\definecolor{ForestGreen}{RGB}{34,106,46}

\lstdefinelanguage{minizinc}{
    morekeywords={
        %% MiniZinc keywords
        %%
        ann, annotation, any, array, assert,
        bool,
        constraint,
        else, elseif, endif, enum, exists,
        float, forall, function,
        if, in, include, int,
        list,
        minimize, maximize,
        of, op, output,
        par, predicate,
        record,
        set, solve, string,
        test, then, tuple, type,
        var,
        where,
        <, <=, >, >=,
        +,-,
        %% MiniZinc functions
        %%
        abort, abs, acosh, array_intersect, array_union,
        array1d, array2d, array3d, array4d, array5d, array6d, asin, assert, atan,
        bool2int,
        card, ceil, combinator, concat, cos, cosh,
        dom, dom_array, dom_size, dominance,
        exp,
        fix, floor,
        index_set, index_set_1of2, index_set_2of2, index_set_1of3, index_set_2of3, index_set_3of3,
        int2float, is_fixed,
        join,
        lb, lb_array, length, let, ln, log, log2, log10,
        min, max,
        pow, product,
        round,
        set2array, show, show_int, show_float, sin, sinh, sqrt, sum,
        tan, tanh, trace,
        ub, and ub_array,
        %% Search keywords
        %%
        bool_search, int_search, seq_search, priority_search,
        %% MiniSearch keywords
        %%
        minisearch, search, while, repeat, next, commit, print, post, sol, scope, time_limit, break, fail
    },
    sensitive=true, % are the keywords case sensitive
    morecomment=[l][\em\color{ForestGreen}]{\%},
    %morecomment=[s]{/*}{*/},
    morestring=[b]",
}

%% Settings for listings
%%
\lstset{ %
    language=minizinc,                  % the language of the code
%    backgroundcolor=\color{lightgray},  % choose the background color; you must add
                                        % \usepackage{color} or \usepackage{xcolor}
    %basicstyle=\ttfamily\small,    % the size of the fonts that are used for the code
    basicstyle=\ttfamily\small,    % the size of the fonts that are
    aboveskip=3mm,
    belowskip=1mm,
    breakatwhitespace=false,            % sets if automatic breaks should only happen at whitespace
    breaklines=true,                    % sets automatic line breaking
    captionpos=b,                       % sets the caption-position to bottom
    commentstyle=\color{ForestGreen},   % comment style
    %deletekeywords={...},              % if you want to delete keywords from the given language
    escapeinside={\%*}{*)},             % if you want to add LaTeX within your code
    extendedchars=true,                 % lets you use non-ASCII characters; for 8-bits
                                        % encodings only, does not work with UTF-8
%    frame=single,                       % adds a frame around the code
    keepspaces=true,                    % keeps spaces in text, useful for keeping indentation
                                        % of code (possibly needs columns=flexible)
    keywordstyle=\ttfamily\color{blue}, % keyword style
%   keywordstyle=\bfseries\color{blue}, % keyword style
    %morekeywords={*,...},              % if you want to add more keywords to the set
    numbers=none,                       % where to put the line-numbers; possible values are (none, left, right)
    %numbersep=5pt,                     % how far the line-numbers are from the code
    %numberstyle=\tiny\color{Gray},     % the style that is used for the line-numbers
    rulecolor=\color{black},            % if not set, the frame-color may be changed
                                        % on line-breaks within not-black text (e.g. comments (green here))
    showspaces=false,                   % show spaces everywhere adding particular
                                        % underscores; it overrides 'showstringspaces'
    showstringspaces=false,             % underline spaces within strings only
    showtabs=false,                     % show tabs within strings adding particular underscores
    %stepnumber=1,                      % the step between two line-numbers. If it's 1, each line will be numbered
    stringstyle=\color{Red},            % string literal style
    tabsize=2,                          % sets default tabsize to 2 spaces
    title=\lstname                      % show the filename of files included with \lstinputlisting;
                                        % also try caption instead of title
}



\newcommand{\dln}{\ensuremath{\mathit{dln}}}


\newcommand{\calP}{{\cal P}}
\newcommand{\real}{\mathbb{R}}
\newcommand{\sol}{\mathsf{sol}}

\def\luisSays#1{{\color{red} Comentario Luis: #1}}
\newcommand{\SPI}{\ensuremath{\mathit{SPI}}}
%%% Local Variables:
%%% mode: latex
%%% TeX-master: "MT_scheduling"
%%% End:

\usepackage{listings}
\usepackage{booktabs}
\usepackage{verbatim}
%\usepackage[svgnames]{xcolor}
\lstdefinelanguage{minizinc}{
    morekeywords={
        %% MiniZinc keywords
        %%
        ann, annotation, any, array, assert,
        bool,
        constraint,
        else, elseif, endif, enum, exists,
        float, forall, function,
        if, in, include, int,
        list,
        minimize, maximize,
        of, op, output,
        par, predicate,
        record,
        set, solve, string,
        test, then, tuple, type,
        var,
        where,
        %% MiniZinc functions
        %%
        abort, abs, acosh, array_intersect, array_union,
        array1d, array2d, array3d, array4d, array5d, array6d, asin, assert, atan,
        bool2int,
        card, ceil, combinator, concat, cos, cosh,
        dom, dom_array, dom_size, dominance,
        exp,
        fix, floor,
        index_set, index_set_1of2, index_set_2of2, index_set_1of3, index_set_2of3, index_set_3of3,
        int2float, is_fixed,
        join,
        lb, lb_array, length, let, ln, log, log2, log10,
        min, max,
        pow, product,
        round,
        set2array, show, show_int, show_float, sin, sinh, sqrt, sum,
        tan, tanh, trace,
        ub, and ub_array,
        %% Search keywords
        %%
        bool_search, int_search, seq_search, priority_search,
        %% MiniSearch keywords
        %%
        minisearch, search, while, repeat, next, commit, print, post, sol, scope, time_limit, break, fail
    },
    sensitive=true, % are the keywords case sensitive
    morecomment=[l][\em]{\%},
    %morecomment=[s]{/*}{*/},
    morestring=[b]",
}
\lstset{language=minizinc,captionpos=b}
\begin{document}

\begin{frontmatter}

%% Title, authors and addresses

%% use the tnoteref command within \title for footnotes;
%% use the tnotetext command for theassociated footnote;
%% use the fnref command within \author or \address for footnotes;
%% use the fntext command for theassociated footnote;
%% use the corref command within \author for corresponding author footnotes;
%% use the cortext command for theassociated footnote;
%% use the ead command for the email address,
%% and the form \ead[url] for the home page:
%% \title{Title\tnoteref{label1}}
\title{Metamorphic testing of scheduling problems with constraints\tnoteref{t1}}
\tnotetext[t1]{This work has been supported by the Spanish Ministry of Science and Innovation and the European Regional Development Fund (ERDF) (projects  RTI2018-093608-BC33 and RED2018-102472-T),
% Luis y Sona
the State Research Agency (AEI) of the Spanish Ministry of Science and Innovation under grant PID2021-122215NB-C31 (AwESOMe) and the Region of Madrid   under grant  S2018/TCS-4314 (FORTE-CM) co-funded by EIE Funds of the European Union.
}

% IST
% JSS
% STVR (Q3)
% IEEE TSE


%% \tnotetext[label1]{}
%% \author{Name\corref{cor1}\fnref{label2}}
%% \ead{email address}
%% \ead[url]{home page}
%% \fntext[label2]{}
%% \cortext[cor1]{}
%% \affiliation{organization={},
%%             addressline={},
%%             city={},
%%             postcode={},
%%             state={},
%%             country={}}
%% \fntext[label3]{}

%% use optional labels to link authors explicitly to addresses:
%% \author[label1,label2]{}
%% \affiliation[label1]{organization={},
%%             addressline={},
%%             city={},
%%             postcode={},
%%             state={},
%%             country={}}
%%
%% \affiliation[label2]{organization={},
%%             addressline={},
%%             city={},
%%             postcode={},
%%             state={},
%%             country={}}


\author[1]{M. Carmen de Castro-Cabrera \corref{cor1}}
\ead{maricarmen.decastro@uca.es}
\author[2]{Sonia Est\'evez-Mart\'{\i}n\corref{cor2}}
\ead{soesteve@ucm.es}
\author[2]{Luis Llana D\'{\i}az\corref{cor3}}
\ead{llana@ucm.es}
\author[3]{Antonio Garc\'{\i}a-Dom\'{\i}nguez}
\ead{a.garcia-dominguez@york.ac.uk}


\cortext[cor1]{Corresponding author}
%\cortext[cor2]{Corresponding author}
%\cortext[cor3]{Corresponding author}

\affiliation[1]{
  organization={Department of Computer Engineering, University of Cadiz},
   country={Spain}}

\affiliation[2]{
  organization={Computing Systems Department, Complutense University of Madrid}, country={Spain}}

\affiliation[3]{
  organization={Department of Computer Science, University of York}, country={UK}}



\begin{abstract}

  \textbf{Context:}

  \textbf{Objective:}

  \textbf{Method:}

  \textbf{Results:}

  \textbf{Conclusion:}

\end{abstract}


\begin{keyword}
%% keywords here, in the form: keyword \sep keyword
 Metamorphic Testing \sep Mutation Testing \sep Scheduling Problems \sep Constraint Programming \sep MiniZinc
\end{keyword}

\end{frontmatter}

\section{Introduction}

Contexto de scheduling problems.
-------- Begin --------
El Resource-Constrained Project Scheduling Problem (RCPSP) \cite{herroelen1998resource, abdolshah2014review,hartmann2022updated}
This problem consist
of A RCPSP consists of resources, tasks, and precedences between some tasks where resources have of a specific capacity and tasks need some capacity of some resource to be executed.
 Here, we consider resources with a constant discrete capacity over time and tasks with a constant discrete duration and resource requirements.
 The objective is to find a optimal schedule with respect to the earliest end time of the schedule where the tasks' resource requirements do not exceed the resource capacities to any time and each precedence is met.


Falta terminar

-------- end --------



%Importancia de probar el resolvedor y/o las reglas.

The main contributions of this paper are:
\begin{itemize}
    \item Testing scheduling programs as a set of constraints implemented in MiniZinc using several software testing techniques.
    \item Design and implementation of metamorphic relations for testing scheduling programs with metamorphic testing.
    \item Design and implementation of mutation operators for MiniZinc, for testing programs in this language using mutation testing.
    \item Application of mutation testing to MiniZinc programs.
    \item Application of metamorphic testing to MiniZinc programs to improve the test case suite.
    \item Analysis of the effectiveness of mutation operators.
    \item Analysis of the effectiveness of implemented metamorphic relations.
\end{itemize}

%Esquema del resto del artículo.
This paper is structured as follows: section 2 describes the background and definitions used in the rest of the paper, section 3 explains the proposal presented, section 4 follows with the methodology used for the experiments, followed by the results obtained and the discussion of the results in section 5. Section 6 describes the threats to validity, and finally, the conclusions and future work.


\section{Background}



\subsection{Scheduling problems in constraint programming}

\textit{Constraint programming} is a paradigm for solving combinatorial problems where relationships between variables are expressed in the form of constraints \citep{rbw06}.
A constraint problem is solved when all the variables have been instantiated and the constraints are satisfied. In general, variables range over a finite set of values, corresponding with \textit{Constraint Programming over Finite Domains} \cite{schulte2006finite}.


\begin{definition}
  A \textit{Constraint Satisfaction Problem}
  \citep{guns2017miningzinc} (CSP) is defined as a triple
  $( X, D, C)$, where $X=(x_{1},\dots ,x_{n})$ is a tuple of $n$
  variables.  $D=D_{1}\times D_{2}\cdots \times D_{n}$, such that
  $D_i$ is the domain of values for each variable $x_i$, and
  $C=(c_{1},\dots ,c_{m})$ are the restrictions that delimit the
  possible values that the variables can take.  Each constraint
  $c_{i}$ is a subset of $D$ that indicates the values that satisfy
  the restriction.
\end{definition}



Specific constraint solvers are used to find the solution of CSP problems, that is, a \textit{constraint solver} is a decision procedure that checks whether a set of constraints can be satisfied.
Constraint solvers perform a systematic exploration of the search space until either a solution is found or the whole tree is explored without success.
Solvers uses classical propagation techniques \citep{van89} combined with {\em backtracking},
a recursive algorithmic that tries to build a solution incrementally, that is, if the partial assignment of the variables is not suitable, then backtrack and try other assignments.


MiniZinc~\cite{DBLP:journals/constraints/MarriottNRSBW08} is a modeling
language that helps to formulate a CSP. The user can specify a CSP by
using typed variables (Boolean, Integer, Float, Set, Array) and
expressions that can use mathematical expressions (sums, products,
comparison, logical operators), looping over sets (forall, exists), etc.
A MiniZinc model can be seen as a program that solve a constraint
problem. These programs (or models) the objects we are going to
consider for testing: Implementation Under Testing (IUT).

Scheduling problems are modeled with certain types of constraints.
Specifically, we deal with the following constraints in this work:
\textit{precedence constraints} where it is specified that a constraint
must be executed before another; and the \textit{resource
  limitations constraints} where certain tasks need to use a series of
resources which are limited and shared. In order to formalize a
scheduling problem we need to consider a finite set of \emph{tasks} that we will
denote $T$, we will denote the elements of $T$ as $t_{1}$ , $t_{2}$,
etc.
We can have one or more processors where the tasks are executed, so we
need a finite set of processors $P$. Some problems do not consider
processors, in these cases we consider that there is only one processor.
Each individual task will have a \emph{duration} that depends on the
processor it is executed, we will
consider a function $D$ that will indicate the duration of any task, a
positive real number.
We will consider an order among the tasks $\prec$ that
will indicate the precedence between tasks: $t_{i}\prec t_{j}$.
Each task need some resources, for instance in the \emph{Moving
  Furniture Problem}, moving a piano needs
3 people and 2 trolleys. Therefore, we need to consider a multiset of
Resources that we will denote $R$,
the multiplicity of any element denotes the number of
instances available for each resource. Then, we need to consider the
resources needed for completing a task, it will be represented by a
function $L$: $L(t)$ is the multiset of resources needed for the task
$t$. Finally, the whole scheduling problem must be completed before a
fixed deadline that we denote $\dln$.

\begin{definition}
  A \emph{scheduling problem instance} is a tuple $i=(T, P, D, \prec, R, L, \dln)$, where
  \begin{itemize}
  \item $T$ is a finite set of tasks. We can assume a numeration of
    the tasks $T=\{t_{1},\ldots t_{n}\}$. For the shake of simplicity
    we will call $T(k)=t_{k}$, for $1\leq k\leq n$.
  \item $P$ is a finite set of processors.
  \item $D$ is a mapping $D:T\times P\mapsto \real^{+}$
    \footnote{$\real^{+}=\{x\ |\ x\in\real,\ x>0\}$}
  \item $\prec\subseteq T\times T$ is a non-reflexive and transitive relation that
    indicates the precedence among tasks.
  \item $R$ is a multiset of resources.
  \item $L: T\mapsto\calP(R)$ is the mapping of associates the
    resources needed for any task.
  \item $\dln\in\real^{+}$ is the deadline of the scheduling problem.
  \end{itemize}
  We denote the set of scheduling problem instances as \SPI.
\end{definition}


Scheduling problem instances are the inputs of the model implemented
in MiniZinc. When executing the model with a particular instance, the
output of MiniZinc gives the minimum time require to complete all the
tasks and schedule.
In this paper, we are going to
consider only the minimum time in which it is possible to
schedule the problem, that is the \emph{makespan}. Sometimes the
scheduling problem has not a solution. In this case, we consider that
the makespan is infinite. To formalize this, we use
the symbol $\infty$. We will denote
$\sol(i)\in\real^{+}\cup\{\infty\}$\footnotemark the solution of the scheduling
problem $S$.

\footnotetext{We assume $x<\infty$ for all $x\in\real$ and $\infty\not<\infty$}

\begin{definition}
Let us consider a scheduling problem.
A \emph{metamorphic relation}  $MR\subseteq\SPI\times\SPI\times(\real^{+}\cup\{\infty\})\times(\real^{+}\cup\{\infty\})$
is a relation over two
scheduling problem instances $i_1$ and $i_2$, and the outputs
corresponding
to the execution of $o_{1}=\sol(i_1)$ and $o_{2}=sol(i_2)$.
$i_1$ is a \emph{source input} and that $i_2$ is a \emph{follow-up input}.
\end{definition}

\subsection{Metamorphic testing}
Metamorphic testing (MT) is a software testing technique that is based in MRs \cite{segura2011automated}.
An MR is a relationship defined between multiple inputs and outputs that is met.
The goal is obtaining (by MRs) new test cases (follow-up test cases named) that detect errors that original test cases did not detect. Chen introduced this name in~\cite{chen1998metamorphic}, it has been applied to numerous contexts and programs. A specific event is held annually~\cite{ 7961643,Xie:2019:3340651} from 2016.
An example of the usefulness of MRs is given as follows. Consider, for example, a program SP, which implements the computation of the shortest path between any two nodes $a$, and $b$, $SP(G,a,b)$. For testing whether the solution is correct, we can obtain another test case, in which the path is in reverse order $SP(G,b,a)$. Therefore, it must be satisfied that $SP(G,a,b)= SP(G,b,a)$. If the results are different, we have detected a failure in the program. This MR can be defined in general terms as:
$MR: \forall x, y \in G, SP(G, x,y) = SP(G, y,x) $
Despite the fact that MT has been applied to multiple domains and programs ~\cite{Chen:2018:MTR:3177787.3143561,almendros2021metamorphic}, its application to programs that implement scheduling problems implemented in MiniZinc constraint language is a novel proposal. Hence, Minizinc has been applied to support MT~\cite{de2019using}.\\


\subsection{Mutation testing}
%% Sección 2.3
%% Referencias mutación:
%%% Review: PAPADAKIS2019275, parsai2literature
%%% Operadores de mutacion: (1996) Offutt96OpMut, Mresa99OpMut, PapadakisM10, puig2010equivalencias
Mutation testing is a well-known and widely applied software testing technique. It enjoys broad acceptance within the academic community, demonstrating a certain maturity, and has been the subject of numerous publications~\cite{parsai2literature, PAPADAKIS2019275}. Essentially, it involves assessing the quality of a test suite by executing them against modified versions of the original program, known as mutants. These mutants are created by injecting common programmer faults into the original code. To generate these mutants, mutation operators are defined, which specify the changes to be made in the program. For example, arithmetic operators can replace one instance of an arithmetic operator with another of the same type (e.g., replacing 'a +' with 'a -', or vice versa). Implementing these mutation operators and generating numerous mutants can be computationally expensive. Consequently, these operators have been extensively studied for various programming languages. Research has shown, among other findings, that a reduced set of these operators can yield good results in most systems~\cite{Offutt96OpMut}.

While these operators target different possible defects, they also incur a cost in the mutation testing process: their mutants need to be generated, and test cases have to be run against them. It is known from the literature that \emph{selective mutation testing} (using only some of the mutants) can have similar testing effectiveness as the full set of mutation operators, for a fraction of the cost~\cite{Mresa99OpMut}.

The literature has also recognised that not all mutants are of the same value: some mutants are killed by most test cases we can come up with, whereas others require very specific test cases, and others are equivalent to the test program. Specifically, Papadakis et al.\ argued against using \emph{subsumed mutants} on experimental studies based on mutation testing, as they could skew the results~\cite{papadakis_threats_2016}. According to Ammann et al.~\cite{ammann_establishing_2014}, mutant $x$ dynamically subsumes mutant $y$ if $x$ has been killed and every test that kills $x$ also kills $y$. These subsumed mutants are redundant for the purposes of designing test cases, and can be safely discarded (e.g. by refining the definition of the mutation operator, or not using it at all).


\section{Proposal}
This proposal aims to evaluate MiniZinc programs, which implement scheduling problems according to a fixed scheme, using two testing methodologies. Firstly, we will employ metamorphic testing, designing and implementing metamorphic relations to generate enhanced test suites. Secondly, we will apply mutation testing. Given the absence of an existing mutation system for MiniZinc, we will develop and implement appropriate mutation operators to create mutants for evaluating the effectiveness of the test suites. Finally, we will assess the efficacy of both the metamorphic relations and the mutation operators.

\subsection{Mutation operators for MiniZinc}
%%%  Hecho***Explicar los operadores por categorías: un apartado para cada categoría: básicos, específicos de MiniZinc y específicos del problema.
%% Hecho: Referencia: buscar revisión sistemática, o referencia primera de Fortran y Mothra

%  Hecho. Listar los operadores, citando como base el conjunto mínimo de Fortran/Mothra, y luego añadiendo cosas específicas de MiniZinc.
%***Operadores de mutación: Carmina
For any mutation system of a programming language, a set of mutation operators is designed and implemented to generate the mutants required for applying this technique. To implement an initial set of operators for MiniZinc, we drew upon existing research, particularly focusing on identifying a sufficient or minimal set of mutation operators~\cite{Offutt96OpMut,Mresa99OpMut,PapadakisM10} to enhance the efficiency of mutation testing. Building upon these studies, we implemented mutation operators for arithmetic, logical, and relational expressions, which are common across most programming languages. Furthermore, as highlighted in~\cite{Mresa99OpMut} it is crucial to consider the language's characteristics and its application context. Accordingly, since MiniZinc is a declarative language (unlike C), we implemented language-specific operators that perform substitutions of logical quantifiers, addressing a potential source of programmer error such as confusing universal and existential quantifiers. Additionally, to account for the specific application context of mutation testing in this case, we focused on MiniZinc programs that implement scheduling problems. These programs often include a \texttt{cumulative}function with multiple parameters, whose incorrect ordering can lead to result errors. Therefore, we implemented a specific operator to detect this type of fault.\\

Consequently, we designed and implemented a total of 43 mutation operators, categorised into five types: arithmetic expressions, relational expressions, logical expressions, quantifiers, and parameter exchange. The relational, logical, and arithmetic operators are fundamental and common to most programming languages. However, we also defined two MiniZinc-specific operator types: quantifier operators and the parameter exchange operator, which targets the \texttt{cumulative} function. Table~\ref{tab:Op-MiniZinc} details the mutation operators for MiniZinc within the context of scheduling problems, classified by these categories.

%%%% Tabla operadores mutación
%% [htb!]
\begin{table*}[htb!]
    \caption{Mutation operators for MiniZinc with scheduling problems}
    \label{tab:Op-MiniZinc}
    \centering
    \small
    \begin{tabular}{l p{1.5\columnwidth}r}
    \toprule
      Operator & \hspace{3cm} Description \\
     \midrule
     \multicolumn{2}{c}{Arithmetic expression operators} \\
     \midrule
     A2S,S2A & Replaces an arithmetic operator (+, -, *, div) with an operator of the same type.\\
     A2M,M2A\\
     A2DV,DV2A\\
     M2S,S2M\\
     M2DV,DV2M\\
     S2DV,DV2S\\
     \midrule
     \multicolumn{2}{c}{Relational expression operators} \\
     \midrule
     LE2LT,LT2LE & Replaces an relational operator $(<, >, <=, >=, ==, !=)$ with an operator of the same type.\\
     LE2GT,GT2LE\\
     LE2NE,NE2LE \\
     LE2GE,GE2LE \\
     LE2EQ,EQ2LE\\
     LT2GT,GT2LT\\
     LT2GE,GE2LT\\
     LT2EQ,EQ2LT\\
     LT2NE,NE2LT\\
     GT2EQ,EQ2GT\\
     GT2NE,NE2GT\\
     EQ2NE,NE2EQ\\

     \midrule
     \multicolumn{2}{c}{Logical expression operators} \\
     \midrule
     C2D y  D2C & Replaces an logical operator ($\wedge$, $\vee$ ) with an operator of the same type.\\
     \midrule
     \multicolumn{2}{c}{Quantifier swap operators} \\
      \midrule
      E2F & Replaces the occurrence of a logical quantifier \texttt{exists}  with a quantifier \texttt{forall}.\\
      F2E & Replaces the occurrence of a logical quantifier \texttt{forall} with a quantifier  \texttt{exists}.\\
     \midrule
      \multicolumn{2}{c}{Parameter swap operator} \\
     \midrule
     CSWAP & Swap parameters 2 and 3 of the MiniZinc \texttt{cumulative} function. \\
    \bottomrule
    \end{tabular}
    \end{table*}

%\begin{comment}
  \begin{table}
    \centering
    \begin{tabular}{lrrr}
    \toprule
    \textbf{Operator} & \textbf{G} & \textbf{E} & \textbf{K} \\
    \midrule
    A2S & 5 & 1 & 4 \\
    LT2NE & 1 & 1 & 0 \\
    \midrule
    % Mutation score = K / (G - E)
    \textbf{Total} & 10 & 2 & 6 \\
    \bottomrule
    \end{tabular}
    \caption{(G)enerated, (E)quivalent, and (K)illed mutant counts before applying metamorphic relations}
    \label{tab:mutant-counts}
\end{table}
\end{comment}

\begin{table}
    \centering
    \begin{tabular}{lrrr}
    \toprule
    \textbf{Operator} & \textbf{G} & \textbf{E} & \textbf{K} \\
    \midrule
%    A2S & 5 & 0 & 5 \\
%    LT2NE & 1 & 1 & 0 \\


A2DV	&	5	&	0	&	5	\\
A2M	    &	5	&	0	&	5	\\
A2S	    &	5	&	0	&	5	\\
C2D	    &	2	&	0	&	0	\\
CSWAP	&	1	&	0	&	1	\\
D2C	    &	1	&	0	&	1	\\
E2F	    &	1	&	0	&	1	\\
EQ2GT	&	1	&	0	&	1	\\
EQ2LE	&	1	&	0	&	0	\\
EQ2LT	&	1	&	0	&	1	\\
EQ2NE	&	1	&	0	&	1	\\
F2E	&	4	&	0	&	4	\\
GT2EQ	&	4	&	0	&	4	\\
GT2LE	&	4	&	0	&	4	\\
GT2LT	&	4	&	0	&	4	\\
GT2NE	&	4	&	0	&	3	\\
LE2EQ	&	4	&	0	&	4	\\
LE2GE	&	4	&	0	&	4	\\
LE2GT	&	4	&	0	&	4	\\
LE2LT	&	4	&	0	&	4	\\
LE2NE	&	4	&	0	&	4	\\
LT2EQ	&	1	&	0	&	1	\\
LT2GE	&	1	&	0	&	1	\\
LT2GT	&	1	&	0	&	0	\\
LT2LE	&	1	&	0	&	1	\\
LT2NE	&	1	&	1	&	0	\\
NE2EQ	&	1	&	0	&	1	\\
NE2GT	&	1	&	0	&	0	\\
NE2LE	&	1	&	0	&	1	\\
NE2LT	&	1	&	0	&	1	\\
S2A	    &	1	&	1	&	0	\\
S2DV	&	1	&	0	&	1	\\
S2M	    &	1	&	0	&	1	\\
   
    \midrule
    % Mutation score = K / (G - E)
    \textbf{Total} & 76 & 2 & 68 \\
    \bottomrule
    \end{tabular}
    \caption{(G)enerated, (E)quivalent, and (K)illed mutant counts before applying metamorphic relations}
    \label{tab:mutant-counts}
\end{table}

Después de esta evaluación, se analizan los operadores que todavía tienen mutantes vivos que son los 6 siguientes.

Se aplican MR3 a estos 6 operadores y se mueren todos.

Operador segunda vuelta	Localización op	MR3


\begin{table}
    \centering
    \begin{tabular}{lrr}
    \toprule
    \textbf{Operator} & \textbf{G} &  \textbf{K} \\
    \midrule
C2D 	& 1	& 1 \\ 
C2D	    & 2	& 1 \\
EQ2LE	& 1	& 1 \\
GT2NE	& 2	& 1 \\
LT2GT	& 1	& 1 \\
NE2GT	& 1	& 1 \\

    \midrule
    % Mutation score = K / (G - E)
    \textbf{Total} & 8 & 8  \\
    \bottomrule
    \end{tabular}
    \caption{(G)enerated and (K)illed mutant counts after applying metamorphic relations}
    \label{tab:mutant-counts2}
\end{table}

\begin{comment}
\subsubsection{Operadores de cuantificadores: E2F y F2E}
E2F: Este operador sustituye la aparición de un cuantificador lógico \textit{Existe}, que en lenguaje MiniZinc es la palabra reservada \texttt{exists}  por un cuantificador \textit{Para todo}, que en lenguaje MiniZinc es la palabra
reservada \texttt{forall}.

F2E: Este operador sustituye la aparición de un cuantificador lógico  \textit{Para todo}, que en lenguaje MiniZinc es la palabra reservada \texttt{forall}  por un cuantificador \textit{Existe}, que en lenguaje MiniZinc es la palabra
reservada \texttt{exists}.

\subsubsection{Operadores relacionales: LE2LT y LT2LE}
LE2LT: Este operador sustituye la aparición de un operador relacional $<=$ por el operador del mismo tipo $<$

LT2LE: Este operador sustituye la aparición de un operador relacional $<$ por el operador del mismo tipo $<=$

\subsubsection{Operadores relacionales: LE2GT y GT2LE}
LE2GT: Este operador sustituye la aparición de un operador relacional $<=$ por el operador del mismo tipo $>$

GT2LE: Este operador sustituye la aparición de un operador relacional $>$ por el operador del mismo tipo $<=$

\subsubsection{Operadores relacionales: LE2NE y NE2LE}
LE2NE: Este operador sustituye la aparición de un operador relacional $<=$ por el operador del mismo tipo $!=$

NE2LE: Este operador sustituye la aparición de un operador relacional $!=$ por el operador del mismo tipo $<=$

\subsubsection{Operadores relacionales: LE2GE y GE2LE}
LE2GE: Este operador sustituye la aparición de un operador relacional $<=$ por el operador del mismo tipo $>=$

GE2LE: Este operador sustituye la aparición de un operador relacional $>=$ por el operador del mismo tipo $<=$

\subsubsection{Operadores relacionales: LE2EQ y EQ2LE}
LE2EQ: Este operador sustituye la aparición de un operador relacional $<=$ por el operador del mismo tipo $==$

EQ2LE: Este operador sustituye la aparición de un operador relacional $==$ por el operador del mismo tipo $<=$

\subsubsection{Operadores relacionales: LT2GT y GT2LT}
LT2GT: Este operador sustituye la aparición de un operador relacional $<$ por el operador del mismo tipo $>$

GT2LT: Este operador sustituye la aparición de un operador relacional $>$ por el operador del mismo tipo $<$

\subsubsection{Operadores relacionales: LT2GE y GE2LT}
LT2GE: Este operador sustituye la aparición de un operador relacional $<$ por el operador del mismo tipo $>=$

GE2LT: Este operador sustituye la aparición de un operador relacional $>=$ por el operador del mismo tipo $<$

\subsubsection{Operadores relacionales: LT2EQ y EQ2LT}
LT2EQ: Este operador sustituye la aparición de un operador relacional $<$ por el operador del mismo tipo $==$

EQ2LT: Este operador sustituye la aparición de un operador relacional $==$ por el operador del mismo tipo $<$

\subsubsection{Operadores relacionales: LT2NE y NE2LT}
LT2NE: Este operador sustituye la aparición de un operador relacional $<$ por el operador del mismo tipo $!=$

NE2LT: Este operador sustituye la aparición de un operador relacional $!=$ por el operador del mismo tipo $<$

\subsubsection{Operadores relacionales: GT2EQ y EQ2GT}
GT2EQ: Este operador sustituye la aparición de un operador relacional $>$ por el operador del mismo tipo $==$

EQ2GT: Este operador sustituye la aparición de un operador relacional $==$ por el operador del mismo tipo $>$

\subsubsection{Operadores relacionales: GT2NE y NE2GT}
GT2NE: Este operador sustituye la aparición de un operador relacional $>$ por el operador del mismo tipo $!=$

NE2GT: Este operador sustituye la aparición de un operador relacional $!=$ por el operador del mismo tipo $>$

\subsubsection{Operadores relacionales: EQ2NE y NE2EQ}
EQ2NE: Este operador sustituye la aparición de un operador relacional $==$ por el operador del mismo tipo $!=$

NE2EQ: Este operador sustituye la aparición de un operador relacional $!=$ por el operador del mismo tipo $==$
%%
\subsubsection{Operadores lógicos: C2D y  D2C}
C2D: Este operador sustituye la aparición de un operador lógico \textit{and}, $\wedge$ , en MiniZinc  por el operador del mismo tipo \textit{or}, $\vee$, en MiniZinc.\\
 D2C: Este operador sustituye la aparición de un operador lógico \textit{or}, $\vee$, en MiniZinc,  por el operador del mismo tipo \textit{and}, $\wedge$ en MiniZinc.

\subsubsection{Operadores lógicos: C2D y  D2C}
C2D: Este operador sustituye la aparición de un operador lógico $\wedge$ (and) por el operador del mismo tipo $\vee$ (or).\\
D2C: Este operador sustituye la aparición de un operador lógico $\vee$ (or) por el operador del mismo tipo $\wedge$ (and).

\subsubsection{Operadores aritméticos: A2S y S2A}
A2S: Este operador sustituye la aparición de un operador aritmético $+$  por el operador del mismo tipo $-$.\\
S2A: Este operador sustituye la aparición de un operador aritmético $-$  por el operador del mismo tipo $+$.

\subsubsection{Operadores aritméticos: A2M y M2A}
A2M: Este operador sustituye la aparición de un operador aritmético $+$  por el operador del mismo tipo $*$.\\
M2A: Este operador sustituye la aparición de un operador aritmético $*$  por el operador del mismo tipo $+$.

\subsubsection{Operadores aritméticos: A2DV y DV2A}
A2DV: Este operador sustituye la aparición de un operador aritmético $+$  por el operador del mismo tipo $div$.\\
DV2A: Este operador sustituye la aparición de un operador aritmético $div$  por el operador del mismo tipo $+$.

\subsubsection{Operadores aritméticos: M2S y S2M}
M2S: Este operador sustituye la aparición de un operador aritmético $*$  por el operador del mismo tipo $-$.\\
S2M: Este operador sustituye la aparición de un operador aritmético $-$  por el operador del mismo tipo $*$.

\subsubsection{Operadores aritméticos: M2DV y DV2M}
M2DV: Este operador sustituye la aparición de un operador aritmético $*$  por el operador del mismo tipo $div$.\\
DV2M: Este operador sustituye la aparición de un operador aritmético $div$  por el operador del mismo tipo $*$.

\subsubsection{Operadores aritméticos: S2DV y DV2S}
S2DV: Este operador sustituye la aparición de un operador aritmético $-$  por el operador del mismo tipo $div$. \\
DV2S: Este operador sustituye la aparición de un operador aritmético $div$  por el operador del mismo tipo $-$.

\subsubsection{Operador de intercambio: CSWAP}
CSWAP: Este operador intercambia los parámetros 2 y 3, de la función de \texttt{cumulative} de Minizinc.
\end{comment}

\subsection{Metamorphic relations for scheduling problems}
\label{sec:def_MT}
%Proponer las reglas metamórficas y justificarlas.


This section exposes the MRs that we propose to detect possible errors
related to modeling scheduling problems: specifically, errors
that are located in precedence and resource sharing
constraints. Each MR is a condition $MR$ that depends
on 4 elements: $i_{1}$ a scheduling problem instance, $i_{2}$ the
following-up scheduling instance that is obtained by modifying
$i_{1}$, $o_{1}$ the output corresponding to $i_{1}$, and $o_{2}$ the
following-up output that is the output of $i_{2}$. A valid
implementation, a model in our case, should make true the MR.

\subsubsection{\mr{1}. Sequentially of all tasks}

Let us consider a scheduling problem instance $i_1 = (T, P, D, \prec, R,
L, \dln)$ such than $|P|=1$ and $\dln\geq\sum_{i=1}^{n}D(t_{i},p_{1})$.
Then we build the scheduling instance problem
$mr_{1}(i_{1})=(T, D, \prec_{2}, R, L, \dln)$ build as follows. Let us
consider $i_{2}=mr_1(i_{1})$, $o_{1}=\sol(i_{1})$, and
$o_{2}=\sol(i_{2})$.
Then, we define $t_{i}\prec_{2} t_{j}$ iff $i<j$, that is, we have a lineal
precedence relation. In these circumstances, if $i_{1}$ has a
solution, that is $o_{1}<\infty$,
then $i_{2}$ has a solution $o_{2}$ such that
$o_{2}=\sum_{i=1}^{|T|}D(t_{i})$.

\begin{framed}
  \begin{displaymath}
      \begin{array}{l}
        \mr{1}(i_1,i_2,o_{1},o_{2}):\\
        \hskip 1em i_1=(T,D,\prec,R,L,\dln)\wedge  \\
        \hskip 1em i_2=mr_1(i_1)\wedge o_{1}=\sol(i_{1})\wedge o_{2}=\sol(i_2)\\
      \hskip 1em \wedge o_{1}<\infty\wedge \dln>=\sum_{i=1}^{|T|}D(T(i)) \\
      \hskip 1em \longrightarrow o_2 = \sum_{i=1}^{|T|}D(T(i))
    \end{array}
  \end{displaymath}
\end{framed}






\subsubsection{\mr{2}. Add inverse precedence}
The idea of this MR is that if we introduce a dependency that
introduce a cycle in a scheduling problem instance with a solution,
then the new problem has no solutions. Formally,
let us consider $i_{1}=(T, P, D, \prec, R, L, \dln)$,
$t_{i}, t_{j}\in T$ such that $t_{i}\prec t_{j}$, and
$o_{1}=\sol(i_{1})<\infty$.
Now let us consider
the modified instance problem as $mr_{2}(i_{1},)=(T, D, \prec_{2}, R, L,
\dln)$,
were $\prec_{2}$ is $\prec$ where the relation $t_{j}\prec t_{i}$ is
added, that is $\prec_{2}= (\prec_{1}\cup \{(t_{j}, t_{i})\})^{+}$.
If we take $i_{2}=mr_{2}(i_{1})$ and $o_{2}=\sol(i_{2})$, then
$o_{2}=\infty$.

\begin{framed}
  \begin{displaymath}
      \begin{array}{l}
    \mr{2}(i_1,i_2,o_{1},o_{2}):\\
      \hskip 1em i_1=(T,P,D,\prec,R,L,\dln)\wedge  \\
      \hskip 1em  i_2=mr_2(i_1)\wedge o_{1}=\sol(i_{1})\wedge o_{2}=\sol(i_2)\\
      \hskip 1em \wedge o_{1}<\infty
      \longrightarrow o_2=\infty
    \end{array}
  \end{displaymath}
\end{framed}


% Si se añade una precedencia que es la inversa de una que ya existe entonces quiere decir que hay un ciclo. Pero, si al añadir esa precedencia inversa sigue dando solución y no es insat, entonces quiere decir que se ha puesto un exists en vez de un forall.

% Obser: El que es insatisfactible tiene un forall y el que tiene solución tiene un exists es el programa mutado (basic-MUT-forall-exists.mzn)
% En este caso, no se ha encontrado un operador definido para este cambio de cuantificadores
% Podemos definirlo, indicando que se cambia un cuantificador lógico por otro (Existencial por universal o viceversa) .

% Esto se hace para matar al mutante que cambia forall por exists.

% Ejemplo correspondiente con el código de la carpeta MR2 del GitHUB y a la Figura \ref{Fig:MR2_Add-inverse-precedence}

% Original: constraint forall(i in PREC)

% Cambiado: constraint exists(i in PREC)

% En los datos se añade la precedencia  SOLDIERS $\rightarrow$ FUNDS para que forme  ciclo con la precedencia que ya existe que es
%  FUNDS $\rightarrow$ SOLDIERS



\subsubsection{\mr{3}. Extreme duration}
The idea of this rule is that if we modify the duration long enough,
the whole scheduling will take longer. Let us consider
$i_{1}=(T, P, D, \prec, R, L, \dln)$, and sum of the duration
of all tasks in all processors and take an upper bound of this sum
$c>\sum_{i=1}^{|T|}\sum_{j=1}^{|P|}D(t_{i},p_{j})$. We modify the
duration of the task $1$ such that its duration is $D(1)+c$, formally, we
define
\begin{displaymath}
  D_{2}(t_{i}, p_{j})=
  \begin{cases}
    D(t_{1}, p_{1}) + c & \text{if } i=1, j=1\\
    D(t_{i}, p_{j}) & \text{otherwise}
  \end{cases}
\end{displaymath}
We take $\dln_{2}=\sum_{i=1}^{|T|}\sum_{j=1}^{|P|}D_{2}(t_{i},p_{j})$.
Then we take $mr_{3}(i_{1})=(T, D_{2}, \prec, R, L, \dln_{2})$. Let us
consider $o_{1}=\sol(i_{1})$, $i_{2}=mr_{3}(i_{1})$ and
$o_{2}=\sol(i_{2})$.
If $o_{1}<\infty$ then $o_{1}<o_{2}<\infty$.

\begin{framed}
  \begin{displaymath}
      \begin{array}{l}
    \mr{3}(i_1,i_2,o_{1},o_{2}):\\
      \hskip 1em i_1=(T,P,D,\prec,R,L,\dln) \wedge  \\
      \hskip 1em  i_2=mr_3(i_1)\wedge o_{1}=\sol(i_{1})\wedge o_{2}=\sol(i_2)\\
      \hskip 1em \wedge o_{1}<\infty
      \longrightarrow o_{1}<o_2<\infty
    \end{array}
  \end{displaymath}
\end{framed}


% Hacemos un mutante donde cambiamos un \texttt{forall} por \texttt{exists},

% \texttt{jobshopMUT-forall-exists-1.mzn}.

% En este caso se aplica el nuevo operador de mutación en el que cambiamos un cuantificador por otro
% y este cambio se puede hacer en otras partes del código dando lugar a nuevos mutantes.







% La Metamorphic Relation \texttt{Extreme Durations}, también mata a otro mutante.

% \texttt{jobshopMUT-and-or.mzn}

% En este caso se ha cambiado la conjunción por la disyunción: \verb+/\  -> \/+

% \subsubsection{cumulative constraint (resources)}



% The cumulative constraint is used for describing cumulative resource usage.

% \texttt{cumulative(array[int] of var int: s, array[int] of var int: d, array[int] of var int: r, var int: b)}

% Schedule for moving furniture:
% It requires that a set of tasks given by start times $s$, durations $d$, and resource requirements $r$, never require more than a global resource bound $b$ at any one time.

% Los códigos son \texttt{Codes/moving.mzn} y \texttt{Codes/moving.dzn}.




\subsubsection{\mr{4}. More resources}
This rule establishes that if we take more resources, the
time required to complete all tasks is, at most, the original one.
Let us consider
$i_{1}=(T, P, D, \prec, R, L, \dln)$, for each $r\in R$ we take a
positive integer $c_{r}$ and then we define
\begin{displaymath}
  R_{2}(r)=R(r)+c_{r}\quad \forall r\in R
\end{displaymath}
Then we take $mr_{4}(i_{1}) = (T, D, \prec, R_{2}, L, \dln)$.
Let us
consider $o_{1}=\sol(i_{1})$, $i_{2}=mr_{4}(i_{1})$ and
$o_{2}=\sol(i_{2})$.
If $o_{1}<\infty$ then $o_{2}\leq o_{1}$.

\begin{framed}
  \begin{displaymath}
      \begin{array}{l}
    \mr{4}(i_1,i_2,o_{1},o_{2}):\\
      \hskip 1em  i_1=(T,P,D,\prec,R,L,\dln) \wedge\\
      \hskip 1em  i_2=mr_3(i_1)\wedge o_{1}=\sol(i_{1})\wedge o_{2}=\sol(i_2)\\
      \hskip 1em \wedge o_{1}<\infty
      \longrightarrow o_{2}\leq o_1
    \end{array}
  \end{displaymath}
\end{framed}




\section{Methodology}
%Listar research questions sobre las propuestas:
In order to evaluate the proposed approach, which combines the two described
software testing techniques (mutation testing and metamorphic
testing),
this section presents several research questions and
describes the program on which the experiments will be conducted.


\subsection{Program under study}
In order to test this novel approach to mutation and metamorphic
testing in constraint programming, we have chosen the program
\lstinline|rcpsp.mzn|, that is a basic version of
the Resource-Constrained Project Scheduling Problem (RCPSP). The
program is taken from the MiniZinc Benchmark
Suite~\footnote{\url{https://github.com/MiniZinc/minizinc-benchmarks/blob/master/rcpsp/rcpsp.mzn}},
and we have included it in \ref{lst:minizinc-rcpsp}.

The inputs have been taken from the MiniZinc Challenge
2013~\footnote{\url{https://github.com/MiniZinc/minizinc-benchmarks/tree/master/rcpsp/data_psplib}}. Due
to the long time required to execute the code, we have selected the simplest
dataset (J30), where all inputs contain 30 tasks.

%%% Local Variables:
%%% mode: LaTeX
%%% TeX-master: "MT_scheduling"
%%% End:



% TO DO: programa bajo prueba? Sonia y Luis

% \emph{explicar qué programas se van a usar para el estudio y por qué}\\
% Aunque hay numerosos ejemplos de programas en MiniZinc que implementan problemas de schedulling, se ha obtado por utilizar un modelo de ejemplo  genérico denominado RCPSP (Resource-Constrained Project Scheduling Problem), que se encuentra en el repositorio \footnote{https://github.com/MiniZinc/minizinc-benchmarks/blob/master/rcpsp/rcpsp.mzn} y que contiene así mismo, numerosos casos de prueba de libre acceso. El listado~\ref{lst:minizinc-rcpsp} muestra el modelo utilizado.




\subsection{Research questions}

\newcommand{\rqoperators}{RQ1}
\newcommand{\rqmetamorphic}{RQ2}
\newcommand{\rqidentifying}{RQ3}

\paragraph{\rqoperators} \emph{Which mutation operators are the most effective for testing constraint solvers?}

Table~\ref{tab:Op-MiniZinc} showed a significant number of mutation operators for the MiniZinc constraint programming language. Some of these operators are generic (being focused on arithmetic and logical operators), whereas some are specific to MiniZinc (E2F and F2E swap \texttt{exists} and \texttt{forall}, and CSWAP specifically targets the MiniZinc \texttt{cumulative} function).

\begin{comment}
% mover a partir de aquí a S2.3. HECHO

While these operators target different possible defects, they also incur a cost in the mutation testing process: their mutants need to be generated, and test cases have to be run against them. It is known from the literature that \emph{selective mutation testing} (using only some of the mutants) can have similar testing effectiveness as the full set of mutation operators, for a fraction of the cost~\cite{Mresa99OpMut}.

The literature has also recognised that not all mutants are of the same value: some mutants are killed by most test cases we can come up with, whereas others require very specific test cases, and others are equivalent to the test program. Specifically, Papadakis et al.\ argued against using \emph{subsumed mutants} on experimental studies based on mutation testing, as they could skew the results~\cite{papadakis_threats_2016}. According to Ammann et al.~\cite{ammann_establishing_2014}, mutant $x$ dynamically subsumes mutant $y$ if $x$ has been killed and every test that kills $x$ also kills $y$. These subsumed mutants are redundant for the purposes of designing test cases, and can be safely discarded (e.g. by refining the definition of the mutation operator, or not using it at all).

% fin de trasfondo para S2.3
\end{comment}

Due to the above reasons, we are interested in identifying the subset of the operators in Table~\ref{tab:Op-MiniZinc} that produce the most useful mutants.
For this purpose, we will use the greedy subsuming mutant identification algorithm from Papadakis et al.~\cite{papadakis_threats_2016}.
It first discards the live and duplicated\footnote{A mutant $x$ is said to be a duplicate of another mutant $y$ if it is killed by the same tests.} mutants from the overall set of mutants $M$.
A subsuming mutant set is then found in a greedy way, by repeatedly adding the mutant which subsumes the most of the remaining mutants and discarding the mutants it subsumes, until no mutants are left.

Starting from a test suite that kills all non-equivalent mutants, we will first apply the algorithm by Papadakis et al.\ and count for each operator the live, duplicated, subsumed, and subsuming mutants.
We have to consider, however, that since this is the first version of the mutation operator set for Mini\-Zinc, there is a high likelihood that we will have a significant number of duplicated mutants that would be discarded but could still provide value for testing.

In order to evaluate the usefulness of the various operators in the presence of duplicated mutants, we propose using a variant of Papadakis' algorithm where for each set of mutants that duplicate each other, a random element will be picked as its sole representative and used to compute subsuming relationships.
Due to the stochasticity of this process, we will run this adapted algorithm 30 times, and then compute three aggregate metrics for each operator: how many times it contributed a subsuming mutant, the size of the smallest set of subsuming operators that it was part of, and the size of the smallest set of subsuming mutants it was part of.
A ``useful'' operator would be one which contributes a subsuming mutant in most if not all of those runs, resulting in a smaller set of operators that produce subsuming mutants, and/or a smaller set of subsuming mutants.

% Se registra el número de mutantes generados en total identificando, los operadores de mutación que los han generado. Es importante analizar también el efecto de los diferentes mutantes generados, y a través de qué operador de mutación se ha generado. Es decir, si unos mutantes mueren con más facilidad o son más difíciles de matar que otros. Cuántos  mueren con el conjunto inicial de casos de prueba y cuántos permanecen vivos, analizando estos últimos, manualmente para detectar posibles mutantes equivalentes, que no se pueden matar. Obtener la tasa de mutación, definida como el cociente entre el número de mutantes muertos y la diferencia entre el total  de mutantes generados y el número mutantes equivalentes.
% Una tasa de mutación del 100\% indica que el conjunto de casos de prueba es de calidad, pues es capaz de matr todos los mutantes.

\paragraph{\rqmetamorphic} \emph{How does combining mutation testing
  with metamorphic testing improve defect coverage? To what extent?
  How much does the test suite improve before and after applying
  metamorphic relations?}

Since metamorphic testing is applied to improve mutation coverage, the
goal is to obtain new test cases that kill mutants whose initial cases
did not, resulting in a more comprehensive and, therefore,
higher-quality test suite. Thus, to answer this question, it is
necessary to measure the improvement that metamorphic testing brings
to the initial suite.

\paragraph{\rqidentifying}
\emph{Are there more effective common features in the metamorphic
  rules for constraint programming, especially for scheduling
  problems?}

We describe the way we have designed the metamorphic rules,
identifying common features in the RCPSP problem within constraint
programming, particularly in MiniZinc. This approach is based on
research into this technique, which reveals the influence of the
program's logic on the design and implementation of the metamorphic
rules.

% Se describe la forma en que se han diseñado las relaciones metamórficas, identificando características comunes de los programas de scheduling y del lenguaje MiniZinc en particular, como lenguaje de restricciones. Este enfoque se basas en las investigaciones y revisiones sistemáticas que se han hecho sobre esta técnica que apuntan a la influencia de la lógica del programa en el diseño e implementación  de las relaciones metamórficas

%* cuál es el comportamiento de los distintos tipos de mutantes propuestos (son difíciles de matar? están subsumidos unos por otros? etc.)?

%* son útiles las relaciones metamórficas para producir nuevas pruebas que matan a los mutantes supervivientes? cuáles de estas reglas son más útiles?

\iffalse

Explicar cómo se van a responder esas research questions: programas MiniZinc seleccionados, conjuntos de pruebas iniciales seleccionados, cuántos mutantes salen de esos programas y cuántos son equivalentes (explicando cómo se detectaron los equivalentes), qué se va a medir (mutantes muertos por el conjunto inicial y por las pruebas producidas por las reglas metamórficas).

\fi



\section{Results and discussion}

Por cada pregunta, se ponen los resultados en bruto, y luego se responden las preguntas originales sobre los resultados.

\subsection{\rqoperators: mutant operator effectiveness}

\begin{table}
    \centering
    \begin{tabular}{lrrrr}
    \toprule
    Operator & \rotatebox{90}{Alive} & \rotatebox{90}{Duplicated} & \rotatebox{90}{Subsumed} & \rotatebox{90}{Subsuming} \\
    \midrule
    % family start
      % row start
            A2S & 0 & 0 & 1 & 4 \\
            S2A & 1 & 0 & 0 & 0 \\
      % row start
            A2M & 0 & 0 & 1 & 4 \\
      % row start
            A2DV & 0 & 2 & 1 & 2 \\
      % row start
            S2M & 0 & 0 & 0 & 1 \\
      % row start
      % row start
            S2DV & 0 & 0 & 1 & 0 \\
    \midrule
    % family start
      % row start
            LE2LT & 0 & 0 & 1 & 3 \\
            LT2LE & 0 & 1 & 0 & 0 \\
      % row start
            LE2GT & 0 & 1 & 1 & 2 \\
            GT2LE & 0 & 4 & 0 & 0 \\
      % row start
            LE2NE & 0 & 1 & 1 & 2 \\
            NE2LE & 0 & 1 & 0 & 0 \\
      % row start
            LE2GE & 0 & 2 & 1 & 1 \\
      % row start
            LE2EQ & 0 & 2 & 0 & 2 \\
            EQ2LE & 0 & 1 & 0 & 0 \\
      % row start
            LT2GT & 0 & 1 & 0 & 0 \\
            GT2LT & 0 & 3 & 1 & 0 \\
      % row start
            LT2GE & 0 & 1 & 0 & 0 \\
      % row start
            LT2EQ & 0 & 1 & 0 & 0 \\
            EQ2LT & 0 & 1 & 0 & 0 \\
      % row start
            LT2NE & 1 & 0 & 0 & 0 \\
            NE2LT & 0 & 1 & 0 & 0 \\
      % row start
            GT2EQ & 0 & 3 & 0 & 1 \\
            EQ2GT & 0 & 1 & 0 & 0 \\
      % row start
            GT2NE & 0 & 2 & 0 & 2 \\
            NE2GT & 0 & 1 & 0 & 0 \\
      % row start
            EQ2NE & 0 & 1 & 0 & 0 \\
            NE2EQ & 0 & 1 & 0 & 0 \\
    \midrule
    % family start
      % row start
            C2D & 0 & 2 & 0 & 0 \\
            D2C & 0 & 0 & 0 & 1 \\
    \midrule
    % family start
      % row start
            E2F & 0 & 1 & 0 & 0 \\
            F2E & 0 & 1 & 2 & 1 \\
    \midrule
    % family start
      % row start
            CSWAP & 0 & 1 & 0 & 0 \\
    \midrule
    \textbf{Total} & 2 & 37 & 11 & 26 \\
    \bottomrule
\end{tabular}

    \caption{Operator statistics, using the unmodified subsuming mutant identification algorithm from Papadakis et al.~\cite{papadakis_threats_2016}}
    \label{tab:opstats-nodups}
\end{table}

33 of the 43 operators in Table~\ref{tab:Op-MiniZinc} were applicable to the program under test (\texttt{rcpsp.mzn}), generating 76 mutants. The starting test suite (provided by the original developers) killed 68 of those mutants, leaving 8 surviving mutants. Later tests produced by the metamorphic relations (analysed in later research questions) killed 6 more, leaving 2 mutants which were manually analyzed and found to be equivalent.

Table~\ref{tab:opstats-nodups} displays the results of applying the original version of Papadakis' subsuming mutant identification algorithm. It shows, for each operator, the number of mutants that survived, that duplicated other mutants (i.e.\ being killed by the same tests), that were subsumed by other mutants, and that were in the subsuming set (i.e.\ mutants which subsume others and are not subsumed by any other mutant). It can be seen that the 26 subsuming mutants were produced by only 13 of the 33 applicable operators. There were very few subsuming mutants outside of the classic arithmetic and relational operators (1 from D2C and 1 from F2E), and the CSWAP operator only produced 1 duplicated mutant. Unfortunately, the fact that Papadakis' algorithm discarded the 37 duplicated mutants prior to detecting subsuming mutants meant that a significant number of subsumption relationships were not considered.

\begin{table}
    \centering
    \begin{tabular}{lrrrr}
    \toprule
    Operator & Runs $\in O$ & $\text{min}(|O|)$ & $\text{min}(|M|)$ \\
    % family start
      % row start
            \midrule
            A2S & 30 & 10 & 40 \\
      % row start
            A2M & 30 & 10 & 40 \\
      % row start
            A2DV & 30 & 10 & 40 \\
      % row start
      % row start
      % row start
    % family start
      % row start
      % row start
            \midrule
            LE2GT & 30 & 10 & 40 \\
            GT2LE & 5 & 11 & 41 \\
      % row start
            LE2NE & 30 & 10 & 40 \\
      % row start
            LE2GE & 30 & 10 & 40 \\
      % row start
            LE2EQ & 30 & 10 & 40 \\
            EQ2LE & 2 & 11 & 41 \\
      % row start
            LT2GT & 7 & 11 & 41 \\
            GT2LT & 14 & 11 & 41 \\
      % row start
      % row start
      % row start
      % row start
            GT2EQ & 30 & 10 & 40 \\
      % row start
            GT2NE & 15 & 10 & 40 \\
            NE2GT & 5 & 11 & 41 \\
      % row start
    % family start
      % row start
            \midrule
            C2D & 10 & 11 & 42 \\
            D2C & 30 & 10 & 40 \\
    % family start
      % row start
    % family start
      % row start
    \bottomrule
\end{tabular}

    \caption{Operator statistics over 30 runs of our variant of Papadakis' subsuming mutant identification algorithm which randomly selects a representative among each set of duplicated mutants. Only operators which produced subsuming mutants in at least one run are shown. $O$ is the set of operators that produced subsuming mutants, and $M$ is the set of subsuming mutants. ``Useful'' operators should appear most of the time in $O$ and be part of smaller families of subsuming operators.}
    \label{tab:opstats-dups}
\end{table}

Table~\ref{tab:opstats-dups} shows the results over 30 runs of our variant of Papadakis' algorithm, which selects randomly one mutant from each set of duplicated mutants as its representative for computing subsuming relationships.
For each run, we compute the set $O$ of operators that produced at least one of the $M$ subsuming mutants, and we record the sizes of both $O$ and $M$.
We can then compute, for each operator, in how many runs they were part of $O$, and what were the minimum sizes of $O$ and $M$ in those runs.
The intuition is that a ``useful'' operator should belong to $O$ in at least most, if not all, of the runs, and that it should be part of a smaller set of operators and/or subsuming mutants.
We can extract several results from this table:

\begin{itemize}
\item There is a core set of 9 mutation operators which were part of $O$ in all runs: A2S, A2M, A2DV, LE2GT, LE2NE, LE2GE, LE2EQ, GT2EQ, and D2C. This suggests they are crucial operators which should always be enabled.

\item The next most useful operator is GT2NE, as it was part of $O$ in 15 of the 30 runs. GT2LT is close (being in $O$ during 14 runs), but GT2NE was part of smaller sets of subsuming operators (up to 10 operators) compared to GT2LT (which required at least 11 operators).

\item C2D would be the fourth most useful operator, being in $O$ during 10 of the 30 runs and requiring the use of at least 11 operators. It did result in the largest sets of subsuming mutants (at least 42), although this is not significantly more than the 40 or 41 of the other operators.
\end{itemize}

In a sense, these results suggest that the traditional arithmetic and relational operators are sufficient, and that we may be even able to limit ourselves to a smaller subset of them.
The logical, quantifier, and operator swapping operators did not contribute significant value beyond those of the traditional operators.

\subsection{\rqmetamorphic: impact of follow-up test cases}

Aunque es un buen resultado el obtenido, se tratará de mejorar aplicando la técnica de pruebas metamórficas, obteniendo nuevos casos de prueba (casos de prueba siguientes) a través de las relaciones metamórficas diseñadas e implementadas.
%\begin{comment}
  \begin{table}
    \centering
    \begin{tabular}{lrrr}
    \toprule
    \textbf{Operator} & \textbf{G} & \textbf{E} & \textbf{K} \\
    \midrule
    A2S & 5 & 1 & 4 \\
    LT2NE & 1 & 1 & 0 \\
    \midrule
    % Mutation score = K / (G - E)
    \textbf{Total} & 10 & 2 & 6 \\
    \bottomrule
    \end{tabular}
    \caption{(G)enerated, (E)quivalent, and (K)illed mutant counts before applying metamorphic relations}
    \label{tab:mutant-counts}
\end{table}
\end{comment}

\begin{table}
    \centering
    \begin{tabular}{lrrr}
    \toprule
    \textbf{Operator} & \textbf{G} & \textbf{E} & \textbf{K} \\
    \midrule
%    A2S & 5 & 0 & 5 \\
%    LT2NE & 1 & 1 & 0 \\


A2DV	&	5	&	0	&	5	\\
A2M	    &	5	&	0	&	5	\\
A2S	    &	5	&	0	&	5	\\
C2D	    &	2	&	0	&	0	\\
CSWAP	&	1	&	0	&	1	\\
D2C	    &	1	&	0	&	1	\\
E2F	    &	1	&	0	&	1	\\
EQ2GT	&	1	&	0	&	1	\\
EQ2LE	&	1	&	0	&	0	\\
EQ2LT	&	1	&	0	&	1	\\
EQ2NE	&	1	&	0	&	1	\\
F2E	&	4	&	0	&	4	\\
GT2EQ	&	4	&	0	&	4	\\
GT2LE	&	4	&	0	&	4	\\
GT2LT	&	4	&	0	&	4	\\
GT2NE	&	4	&	0	&	3	\\
LE2EQ	&	4	&	0	&	4	\\
LE2GE	&	4	&	0	&	4	\\
LE2GT	&	4	&	0	&	4	\\
LE2LT	&	4	&	0	&	4	\\
LE2NE	&	4	&	0	&	4	\\
LT2EQ	&	1	&	0	&	1	\\
LT2GE	&	1	&	0	&	1	\\
LT2GT	&	1	&	0	&	0	\\
LT2LE	&	1	&	0	&	1	\\
LT2NE	&	1	&	1	&	0	\\
NE2EQ	&	1	&	0	&	1	\\
NE2GT	&	1	&	0	&	0	\\
NE2LE	&	1	&	0	&	1	\\
NE2LT	&	1	&	0	&	1	\\
S2A	    &	1	&	1	&	0	\\
S2DV	&	1	&	0	&	1	\\
S2M	    &	1	&	0	&	1	\\
   
    \midrule
    % Mutation score = K / (G - E)
    \textbf{Total} & 76 & 2 & 68 \\
    \bottomrule
    \end{tabular}
    \caption{(G)enerated, (E)quivalent, and (K)illed mutant counts before applying metamorphic relations}
    \label{tab:mutant-counts}
\end{table}

Después de esta evaluación, se analizan los operadores que todavía tienen mutantes vivos que son los 6 siguientes.

Se aplican MR3 a estos 6 operadores y se mueren todos.

Operador segunda vuelta	Localización op	MR3


\begin{table}
    \centering
    \begin{tabular}{lrr}
    \toprule
    \textbf{Operator} & \textbf{G} &  \textbf{K} \\
    \midrule
C2D 	& 1	& 1 \\ 
C2D	    & 2	& 1 \\
EQ2LE	& 1	& 1 \\
GT2NE	& 2	& 1 \\
LT2GT	& 1	& 1 \\
NE2GT	& 1	& 1 \\

    \midrule
    % Mutation score = K / (G - E)
    \textbf{Total} & 8 & 8  \\
    \bottomrule
    \end{tabular}
    \caption{(G)enerated and (K)illed mutant counts after applying metamorphic relations}
    \label{tab:mutant-counts2}
\end{table}
\begin{table}
    \centering
    \begin{tabular}{lrrr}
    \toprule
    \textbf{Operator} & \textbf{G} & \textbf{E} & \textbf{K} \\
    \midrule
A2DV    &       5       &       0       &       5       \\
A2M         &   5       &       0       &       5       \\
A2S         &   5       &       0       &       5       \\
C2D         &   2       &       0       &       0       \\
CSWAP   &       1       &       0       &       1       \\
D2C         &   1       &       0       &       1       \\
E2F         &   1       &       0       &       1       \\
EQ2GT   &       1       &       0       &       1       \\
EQ2LE   &       1       &       0       &       0       \\
EQ2LT   &       1       &       0       &       1       \\
EQ2NE   &       1       &       0       &       1       \\
F2E      &      4       &       0       &       4       \\
GT2EQ   &       4       &       0       &       4       \\
GT2LE   &       4       &       0       &       4       \\
GT2LT   &       4       &       0       &       4       \\
GT2NE   &       4       &       0       &       3       \\
LE2EQ   &       4       &       0       &       4       \\
LE2GE   &       4       &       0       &       4       \\
LE2GT   &       4       &       0       &       4       \\
LE2LT   &       4       &       0       &       4       \\
LE2NE   &       4       &       0       &       4       \\
LT2EQ   &       1       &       0       &       1       \\
LT2GE   &       1       &       0       &       1       \\
LT2GT   &       1       &       0       &       0       \\
LT2LE   &       1       &       0       &       1       \\
LT2NE   &       1       &       1       &       0       \\
NE2EQ   &       1       &       0       &       1       \\
NE2GT   &       1       &       0       &       0       \\
NE2LE   &       1       &       0       &       1       \\
NE2LT   &       1       &       0       &       1       \\
S2A         &   1       &       1       &       0       \\
S2DV    &       1       &       0       &       1       \\
S2M         &   1       &       0       &       1       \\

    \midrule
    % Mutation score = K / (G - E)
    \textbf{Total} & 76 & 2 & 68 \\
    \bottomrule
    \end{tabular}
    \caption{(G)enerated, (E)quivalent, and (K)illed mutant counts before applying metamorphic relations}
    \label{tab:mutant-counts}
\end{table}

%Después de esta evaluación, se analizan los operadores que todavía tienen mutantes vivos que son los 6 siguientes.
%Se aplican MR3 a estos 6 operadores y se mueren todos.

% Operador segunda vuelta       Localización op MR3 ¿Qué quiere decir?
Al aplicar MR3 al conjunto de casos de prueba inicial, se obtiene un conjunto de casos de pruebas siguientes, que se ejecutan contra los 6 mutantes que aún quedan vivos, obteniéndose como resultado que estos casos de prueba matan a los mutantes restantes, por lo que se consigue matar al 100\% de los mutantes, como se puede observar en la tabla~\ref{tab:mutant-counts}.

\subsection{\rqidentifying: features of mutation-effective metamorphic relations}
The definition and explanation of the metamorphic rules are in
Section\ref{sec:def_MT}. In this section, we begin how each
metamorphic rule affects to each part or the \lstinline|rcpsp.mzn|
listed in~\ref{lst:minizinc-rcpsp}. The constraints of the programs
have been divided into 4 groups:
\begin{enumerate}
\item\label{item:precdence} Precedence constraints.
\item\label{item:redundant} Redundant non-overlapping constraints.
\item\label{item:cumulative} Cumulative resource constraints.
\item\label{item:makespan} Makespan constraints.
\end{enumerate}

Según los follow-up tests de las MRs la más efectiva es la \mr{3} que mata a 63, después \mr{4} a 48. Después compreobamos que las \mr{1} y \mr{2} son poco eficientes pues no matan ni a la mitad de los mutantes. 
\mr{1} mata a 24 mutantes y  \mr{2}  a 6.

The metamorphic rule \mr{1} only detects defects in the precedence, redundant non-overlapping and cumulative resource constraints.
\mr{2} solo tiene efecto sobre Precedence constraints, mientras que \mr{3} y \mr{4} tienen efecto en los 4 grupos. 





%%% Local Variables:
%%% mode: LaTeX
%%% TeX-master: "MT_scheduling"
%%% End:


\begin{table}
    \centering
    \begin{tabular}{lrr}
    \toprule
    \textbf{Operator} & \textbf{G} &  \textbf{K} \\
    \midrule
C2D         & 2 & 2 \\
EQ2LE   & 1     & 1 \\
GT2NE   & 1     & 1 \\
LT2GT   & 1     & 1 \\
NE2GT   & 1     & 1 \\

    \midrule
    % Mutation score = K / (G - E)
    \textbf{Total} & 6 & 6  \\
    \bottomrule
    \end{tabular}
    \caption{(G)enerated and (K)illed mutant counts after applying metamorphic relations}
    \label{tab:mutant-counts2}
\end{table}


\section{Threats to validity}
% https://link.springer.com/content/pdf/10.1007/978-3-662-69306-3.pdf
%Threats to validity: internal and external.
%Wohlin S14.4.4.
In the following, we discuss the threats to the validity of the results of the experiments presented in the
previous section.

\subsection{Internal validity}
Mutation operators were designed and implemented for mutation testing, following the recommendations of recent research in this area. Furthermore, improvement algorithms were applied, yielding promising results. Nonetheless, given that this is the initial set of mutation operators, further investigation is warranted to refine them. This may involve the development of novel mutation operators that are more pertinent to the problem domain or the characteristics of the programming language.

Concerning metamorphic testing, metamorphic relations were developed, taking into account the problem context, to generate additional test cases that enhance the quality of the initial test suite, thus meeting the defined objective. However, there is potential for the design of further metamorphic relations that generate new test cases by considering alternative factors, and for the iterative application of existing metamorphic relations to previously generated test cases.

\subsection{External validity}



\section{Conclusion and future work}

Resumen de lo presentado, y de las respuestas a las preguntas.

Planes futuros.

\section*{Acknowledgments}

Reconocimientos de proyecto.

\bibliographystyle{elsarticle-num}
\bibliography{biblio}
\appendix
\onecolumn
\section{Minizinc code}\label{lst:minizinc-rcpsp}
\begin{lstlisting}[caption={Minizic Code for the RSPCP}]
% Model parameters.

  % Resources
    int: n_res;                     % The number of resources
    set of int: Res = 1..n_res;     % The set of all resources
    array [Res] of int: rc;         % The resource capabilities

  % Tasks
    int: n_tasks;                           % The number of tasks
    set of int: Tasks = 1..n_tasks;         % The set of all tasks
    array [Tasks]      of int       : d  ;  % The task durations
    array [Res, Tasks] of int       : rr ;  % The resource requirements
    array [Tasks]      of set of int: suc;  % The task successors

  % Planning horizon
    int: t_max = sum(i in Tasks)(d[i]);     % End time of the planning horizon
    set of int: Times = 0..(t_max - 1);     % Possible start times

% Model variables.

    array [Tasks] of var Times: s;  % The start times
    var 0..t_max: objective      ;  % The project duration (makespan)

% Constraints.

  % Precedence constraints
    constraint forall ( i in Tasks, j in suc[i] )
       ( s[i] + d[i] <= s[j] );

  % Redundant non-overlapping constraints
    constraint forall ( i, j in Tasks where i < j )
        (if exists(r in Res)(rr[r, i] + rr[r, j] > rc[r]) then
                s[i] + d[i] <= s[j]   \/ s[j] + d[j] <= s[i]
            else
                true
            endif
        );

  % Cumulative resource constraints
    constraint forall ( r in Res )
        (  let { set of int: RTasks = { i | i in Tasks
                                where rr[r, i] > 0 /\ d[i] > 0 },
                int: sum_rr = sum(i in RTasks)(rr[r, i])
            } in (if RTasks != {} /\ sum_rr > rc[r] then
                    cumulative(
                        [ s[i] | i in RTasks ],
                        [ d[i] | i in RTasks ],
                        [ rr[r, i] | i in RTasks ],
                        rc[r]
                    )
                else
                    true
                endif
            )
        );

  % Makespan constraints
    constraint forall ( i in Tasks where suc[i] == {} )
       ( s[i] + d[i] <= objective );

% Objective.
    solve :: int_search( s ++ [objective], smallest, indomain_min, complete )
       minimize objective;
\end{lstlisting}

%%% Local Variables:
%%% mode: LaTeX
%%% TeX-master: "MT_scheduling"
%%% End:

\end{document}

%%% Local Variables:
%%% mode: latex
%%% TeX-master: t
%%% End:
